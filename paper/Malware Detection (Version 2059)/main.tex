\documentclass{article}

\usepackage[english]{babel}
\usepackage{numprint}
\usepackage{amsmath}
\usepackage{bm}
\usepackage{graphicx}
\usepackage[colorlinks=true, allcolors=blue]{hyperref}
\usepackage{booktabs}
\usepackage{amssymb}
\usepackage{float}
\usepackage{xfrac}


\newcommand{\barra}{\:\Big|\:}

% Set page size and margins
% Replace `letterpaper' with`a4paper' for UK/EU standard size
\usepackage[letterpaper,top=2cm,bottom=2cm,left=3cm,right=3cm,marginparwidth=1.75cm]{geometry}


\title{Your Paper}
\author{You}

\begin{document}
\maketitle

\begin{abstract}
Your abstract.
\end{abstract}

\section{Introduction}

\section{Environment}

The testing environment is composed by packet capture files provided by the Stratosphere Project. These so called \textit{pcaps} are the network traffic generated by a computer which is known to be malware infected or not. We consider only the DNS packets, ignoring all the other information.\\
The panoramic of the available \textit{pcaps} can be seen in Table \ref{tab:pcaps}.

\begin{table}[H]
    \centering
    \begin{tabular}{l|llll}
         infected & number & DNS packets & Unique domain names & Different malwares \\
         \midrule
         no & 17 & \numprint{25334033} & \numprint{94215} & - \\
         yes & 33 & \numprint{596768} & \numprint{16214} & 20 \\
    \end{tabular}
    \caption{\textbf{number}: it describes the number of different captures. Each capture consists of a different Operative System (which is always Windows), has different length in duration and size,  \textbf{DNS packets}: the DNS packets number considering all the pcaps. \textbf{Unique domain names}: the number of different Domain Names considering all the pcaps. \textbf{Different malwares}: the number of \textit{different} malwares used for infection; one malware may have infected  multiple times. }
    \label{tab:pcaps}
\end{table}

% a scatter plot showing each pcap with color=malware, size=size, x=duration, y=uniques.
% not very useful to understand pcaps distribution



\subsection{Malwares}

The $33$ not-healthy \textit{pcaps} are infected by 20 different malwares, listed in Table \ref{tab:malwares}. The few information obtained for each of them come from VirusTotal, using the hash provided by each pcap. It is important to understand that these malwares uses different DGAs families and some of them seem to not use a DGA at all.


\subsubsection{DGA values}

For each malware, we manually assigned a value $\omega \in \{ 0, 1, 2, 3 \}$ such that:
\begin{itemize}
    \item $\omega=1$: the malwares does not seem to generate DGA domains.
    \item $\omega=2$: the malware behaves as if it uses a DGA but generates a very small number of domains.
    \item $\omega=3$: the malware most likely uses a DGA.
\end{itemize}
%
These values have been decided by inspecting all the DNs obtained from all the pcaps infected by each different malware. Then we make three views:
\begin{enumerate}
    \item we filter out those DNs which do not belong to the top 10 milion domain lists.
    \item like the previous one but we filter out all those DNS requests having response nxdomain greater than 0.
    \item we search for DGA generated domains that use Dynamic DNS providers removing the top10milion filter and sorting by the base domain name number of appearances.
\end{enumerate}


\begin{table}[H]
    \centering
\begin{tabular}{lllr}
    name & year & binary &  pcap infected \\
\midrule
                PWS:Win32/Zbot!GO & 2013 &                            8219s.exe &      1 \\
         Backdoor:Win32/Caphaw.AH & 2014 &                           nltool.exe &      1 \\
          Trojan:Win32/Tiggre!rfn & 2017 &                            Setup.exe &      2 \\
                PWS:Win32/Zbot!GO & 2013 &                           3ab45z.exe &      2 \\
                PWS:Win32/Zbot!GO & 2013 &                           FzPfH6.exe &      1 \\
              Trojan:Win32/Necurs & 2016 &                          sample1.exe &      1 \\
            Trojan:Win32/Matsnu.R & 2014 &                         WINAPI32.EXE &      1 \\
                PWS:Win32/Zbot!GO & 2014 &                             yL0T.exe &      2 \\
       Trojan:Win32/Skeeyah.A!bit & 2017 & ebb20174ee893c0754654668f3e837ff.exe &      1 \\
TrojanDownloader:Win32/Dofoil!rfn & 2017 &                             java.exe &      1 \\
            Trojan:Win32/Emotet.G & 2015 &                   Ferer.exe/9049.exe &      1 \\
                Worm:Win32/Netsky & 2006 &                             data.pif &      2 \\
    Trojan-Banker.Win32.Tinba.hrd & 2016 &                Mezzaluna calante.exe &      1 \\
            Trojan:Win32/Emotet.G & 2015 &                            Ferer.exe &      1 \\
           Backdoor:Win32/Htbot.B & 2015 &                            Htbot.exe &      2 \\
       TrojanProxy:Win32/Bunitu.K & 2015 &                      troyanproxy.exe &      1 \\
           Trojan:Win32/Bulta!rfn & 2013 &                           oleprn.dll &      1 \\
            Trojan:Win32/Miuref.R & 2016 &                    Postmodernism.exe &      6 \\
   TrojanDownloader:Win32/Obvod.M & 2013 &                              105.exe &      4 \\
                   PWS:Win32/Zbot & 2014 &            salesforce\_ssl\_cert.zip &      1 \\
\bottomrule
\end{tabular}
    \caption{Malwares panoramic.}
    \label{tab:malwares}
\end{table}

\section{LSTM neural network}
We use a LSTM neural network with 128 units trainined with Dataset.


\subsection{25-DGA Dataset}

Our dataset is the one proposed in \cite{univpm}, built by extracting a set of domains generated by 25 different DGA families from the Netlab Opendata Project repository and using Alexa as an authoritative source for benign domain names.
In total, we collected \numprint{675000} domain names: 50\% of the domains are DGA-based and 50\% are Alexa domains. The \numprint{337500} malicious DNs are equally distributed among the 25 DGA families, thus we have \numprint{13500} domains for each family. Among such DGAs, 13 are time-dependent algorithms and 12 are time-independent ones.

\begin{table}[b]
    \centering
    \begin{tabular}{lll}
        Time-Dependent DGAs & \# Domains & Examples \\
        \midrule
        conficker & \numprint{13500} & akdtfkh.com.bs, flqxktto.co.za, hiroulyv.mu \\
        corebot & \numprint{13500} & a2ix16edy61f1xg.ddns.net, 1ruvwjy4u850385.ddns.net \\
        cryptolocker & \numprint{13500} & ymirohscgjxm.ru, ugjywfellfju.co.uk \\
        emotet & \numprint{13500} & vtypqtkjnkyfycwj.eu, byoljbjregchfbtd.eu \\
        gozi & \numprint{13500} & animargumenta.com, surfacepapanobi.com \\
        matsnu & \numprint{13500} & branch-tower.com, film-water-image.com \\
        murofet & \numprint{13500} & nirauhusfeormfuwdovrrfinj.biz, jxqtupnmsmknn.info \\
        necurs & \numprint{13500} & wjilrilcim.ru, gfqcpaifkxd.bit \\
        nymaim & \numprint{13500} & smilefavorite.uz, soft-professor.com \\
        padcrypt & \numprint{13500} & adcaeccnacfnccma.com, bnmfladdaccalkff.com \\
        qadars & \numprint{13500} & 8igi8qwu468u.com, e7wx2jkl2zw5.net \\
        suppobox & \numprint{13500} & groupfirst.net, brooklynnwashington.net \\
        symmi & \numprint{13500} & qaliesvomo.ddns.net, meakugulu.ddns.net \\
        \midrule
        dircrypt & \numprint{13500} & zqmyueidka.com, rafspijepwfkwna.com \\
        fobber & \numprint{13500} & guzhbhepwd.com, dxpgomjalm.com \\
        kraken & \numprint{13500} & gqnysr.dyndns.org, qqvgpzewpaj.com \\
        pushdo & \numprint{13500} & suxqeaki.ru, hilgeaki.ru \\
        pykspa & \numprint{13500} & cgyooueoya.biz, hhqaiifox.com \\
        ramdo & \numprint{13500} & ciqauuaieywkkams.org, eiecsgkuamkwscyg.org \\
        ramnit & \numprint{13500} & aimbvmwceer.com, temhosjjrcini.com \\
        ranbyus & \numprint{13500} & aesxwvhepgpsglydp.su, ibcavldgocpkyegbq.cc \\
        rovnix & \numprint{13500} & 46gfypffuuasknujay.ru, vscfnykbbbe4h2wx5y.com \\
        simda & \numprint{13500} & hacudaz.su, wycinep.net \\
        tinba & \numprint{13500} & impxyrifhmld.biz, hwwjgnhhryry.com \\
        vawtrak & \numprint{13500} & fusirnat.com, maduhgumde.com \\
        \midrule
        Alexa & \numprint{337500} & perrosguru.ru, nhst.no, americanphotomag.com \\
\end{tabular}
    \caption{25 DGA families and Alexa composing the Dataset.}
    \label{tab:dataset}
\end{table}


\subsection{Pre-processing}

Once we chose the 25-DGA Dataset to train the neural network, we have multiple possible training dataset depending on the part of the \textit{domain name} we want to use for the training, or equivalently, how we decide to treat the fixed number of a domain, the \textbf{suffix}.

% Each domain name is composed by:
% \begin{itemize}
%     \item the top level domain (TLD), the last portion of the domain name which must be part of the official list of TLDs maintained by the Internet Assigned Numbers Authority (IANA), for example, org, com, eu, and edu. Currently, \numprint{1547} valid TLDs are listed in the root zone.
%     \item the public suffix, which includes both TLDs as well as suffixes. We used the Mozilla Foundation Public List which contains more than \numprint{11000} valid public suffixes.
% \end{itemize}


% \subsection{TLD confusion}

There is a bit of confusion about Top-Level-Domain definition. It's original definition is the first label from the right of the domain name, and a list of this type of TLDs is maintained by IANA \cite{iana_link} and counts \numprint{1487} elements. \\
With the evolving of internet, the original TLDs list has grown, and the TLDs definition changed: nowadays it can consist of more than the label on the right and can also be referred to as \textit{effective TLD} or \textit{suffix} \cite{psl_wiki}. %
There is no standard definition about that, but practically domain registrars are using and defining this standard nowadays. \\
For example, consider the \texttt{.uk} and \texttt{.co.uk} TLD extensions: private domains under \texttt{co.uk} must not be considered at all (sub)domains sharing the same \texttt{.uk} TLD. In other words, \texttt{.uk} and \texttt{.co.uk} are two different TLDs.\\
Practically, browser cookies accessibility is defined as all the subdomains of the private domain that created them. For example, if we have cookies generated by \texttt{example.co.uk} and we think that \text{.uk} is the TLD, then these cookies will be accessible by \texttt{*.co.uk} domains, considering \texttt{co.} the private domain, and \texttt{example.} the subdomain. But if we know that \texttt{*.co.uk} is the TLD, than \texttt{example.} will be considered the private domain and the cookies will be accessible just for \texttt{(*.)example.co.uk}, and not for \texttt{(*.)co.uk}.\\

There are two main types of eTLDs:
\begin{itemize}
    \item ICANN: related to geographic regions or public-like institutions.
    \item Private domains: related to private organizations.
\end{itemize}
%
Since \textit{suffixes} cannot be chosen and are defined by Domain Registrars, Domain Generation Algorithms cannot generate this part of the domain name. Furthermore, the training can give too much importance to this part of the domain name if the dataset domain names are unbalanced towards some \textit{suffixes}.
%
Therefore we trained different neural networks with edited version of the original datasets where each domain name has been transformed depending on the part of domain name we chose.

% Please add the following required packages to your document preamble:
% \usepackage{graphicx}
\begin{table}[H]
\centering
\resizebox{\textwidth}{!}{%
\begin{tabular}{llllllllll}
                &\phantom{ab}& \multicolumn{2}{c}{pcaps} &\phantom{ab}& \multicolumn{5}{c}{datasets}      \\
                &&              &            && \multicolumn{3}{l}{count} & \%      &         \\
labels          && count        & \%         && dga     & legit  & both   & dga     & legit   \\
\cmidrule{1-1} \cmidrule{3-4} \cmidrule{6-10}
1               && 325830       & 1.26\%     && 0       & 0      & 0      & -       & -       \\
2               && 17531222     & 67.61\%    && 295710  & 290100 & 585810 & 50.48\% & 49.52\% \\
3               && 5140972      & 19.83\%    && 41790   & 45209  & 86999  & 48.04\% & 51.96\% \\
4               && 1764878      & 6.81\%     && 0       & 2084   & 2084   & -       & -       \\
5               && 1032919      & 3.98\%     && 0       & 5      & 5      & -       & -       \\
\textgreater{}6 && 134980       & 0.51\%     && 0       & 0      & 0      & -       & -       \\
\midrule
total           && 25930801     &            && 337500  & 337398 & 674898 &         &        
\end{tabular}%
}
\caption{}
\label{tab:labels-distribution}
\end{table}


\begin{table}[H]
    \centering
    \begin{tabular}{lllll}
        type && category && examples\\\midrule
        \texttt{icann} && country-code && it $\cdot$ lazio.it \\
        && && uk $\cdot$ co.uk \\
        && && no $\cdot$ flesberg.no \\
        \texttt{icann} && sponsored && aeroclub.aero $\cdot$ zoology.museum $\cdot$ gov $\cdot$ edu \\
        \texttt{icann} && infrastructure && arpa \\
        \texttt{icann} && generic-restricted && biz $\cdot$ name $\cdot$ pro \\
        \texttt{icann} && generic && com $\cdot$ info $\cdot$ mobi $\cdot$ net \\
        \texttt{private} && Amazon && ap-northeast-1.elasticbeanstalk.com \\
        && && s3-ap-northeast-1.amazonaws.com \\
        && && cloudfront.com \\
        
        \texttt{private} && Azure && eastus2.azurestaticapps.net \\
	    \texttt{private} && Other && dyndns-at-home.com \\ 
	    && && wixsite.com	\\
    \end{tabular}
    \caption{Caption}
    \label{tab:my_label}
\end{table}








\section{Notation}

\subsection{Sets definition}

Each DNS packet is defined by many fields, in our discussion we take into account just few of them. We also add the inference value which is a computed field.\\
A DNS packet is defined as:
\[
    d = \left( i, d, r, N(d) \right) = \left( i, d, r, \varepsilon \right)
\]
where:
\begin{itemize}
    \item $i$ is the packet sequence number,
    \item $h$ is the domain name value,
    \item $r$ is the response code,
    \item $N(h)=\varepsilon$ is the inference value (used values of $r$ are listed in table \ref{tab:rcodes}).
\end{itemize}
A DNS packet list is a set of DNS packet defined as:
\[
    \bm{l} = \left\{ d_i \barra 0 \le i < n \right\}
\]
where $n$ is the number of DNS packets of the set. \\
If $\bm{l}$ has been originated from the same capture session (packets comes from the same network flow for a contiguous period of time) then $\bm{l} \rightarrow \bm{p}$ is called a Packet Capture (\texttt{pcap}) having this property: $p_{i+1}$ follows $p_i$ in the capture. In that case, $i$ not only indicates the packet index in the list but also the capture frame number.

\subsubsection{Inference value}

It is the output of the neural network, having as input a domain name.

\begin{equation}
    N(h) = \varepsilon \in [0,1]
\end{equation}

where:
\begin{itemize}
    \item 0 indicates that the domain name is probably not-DGA generated.
    \item 1 indicates that the domain name is probably DGA generated.
\end{itemize}

\subsection{Windowing}
We also define \texttt{pcap} \textit{windowing}, which given a fixed size $s$ permits to split $\bm{p}$ into \textit{windows} of size $s$:
\[
    w_i = \left\{ d_j \barra i \cdot s \le j < n' \right\} \text{ where } n' = \begin{cases} n & \text{if } (i+1) \cdot s > n\\ (i+1) \cdot s & \text{otherwise}\end{cases}
\]
where $n$ is the number of DNS packets of the pcap. The packets $p_{i_j} \in w_i$ are contiguous respective to the capture.
A packet can also be written as:
\[
    \bm{p} = \left\{ w_i \barra 0 \le i < n^w \right\} \text{ where } n^w = \lceil n / s \rceil
\]
that is the concatenation of its windows.

\subsubsection{Windowing regarding requests and responses}
See section \ref{sec:window-type}.

\begin{table}[b]
    \centering
    \begin{tabular}{lll}
    symbol & definition & description \\\midrule
        $p$ & $\left( i, d, r, \varepsilon \right)$ & DNS packet  \\
        $\bm{l}$ & $\left\{ p_0, p_1, ..., p_{n-1} \right\}$ & DNS packet list \\
        $\bm{p}$ & $\left\{ p_0, p_1, ..., p_{n-1} \right\}$ & DNS packet list from the same source \\
        $\bm{w^s_i}$ & $\left\{ p_{s\cdot i}, p_{s\cdot i + 1}, p_{s\cdot i + 2}, ..., p_{s\cdot (i+1) -1}\right\}$ & DNS packet window of size $s$ \\
    \end{tabular}
    \caption{DNS packet sets.}
    \label{tab:DNS-sets}
\end{table}


\subsubsection{Packet functions}
We now define packet functions used to extract useful information from DNS packets:
\begin{itemize}
    \item The \textit{logit function}:
    \[
            \lambda(d) = \log \left(\frac{\varepsilon}{1-\varepsilon}\right)
    \]
    \item The \textit{counter function}, transform the continuous space of $\varepsilon$ into a discrete one:
    \[
        c(d, \, th) = \begin{cases} 1 & \text{if } \varepsilon > th, \\ 0 & \text{otherwise.}\end{cases}
    \]
    where $th \in (0,1)$ indicates an arbitrary threshold value.
    \item The \textit{nx function}, return true if the return code is NXDOMAIN.
    \[
        \chi(p) = \begin{cases} 1 & \text{if } r = 3, \\ 0 & \text{otherwise.} \end{cases}
    \]
\end{itemize}

\subsubsection{List function}
The input of these type of functions is a DNS packet list. We define:
\begin{itemize}
    \item \textit{LLR function}, which uses the \textit{logit function}:
    \[
        \Lambda(\bm{l}) = \sum_{i=0}^{n} \lambda(p_i)
    \]
    \item \textit{counter function}:
    \[
        C(\bm{l}, th) = \sum_{i=0}^{n} c(p_i)
    \]
    \item \textit{nx function}:
    \[
        X(\bm{l}) = \sum_{i=0}^{n} \chi(p_i)
    \]
\end{itemize}
These functions are nether less the sum of the respective packet functions.

% \subsubsection{Detection function}

% We define a detection function that predict if the DNS list is infected or not according to a list function and a threshold:
% \[
%  D(\bm{l}, f, th) = 
%         \begin{cases}
%         1 & \text{if } f(\bm{l}) < th \\
%         0 & \text{if } f(\bm{l}) \ge th \\
%         \end{cases}
% \]
% where $f$ is a list function. Values $1$ and $0$ indicates that $\bm{l}$ is respectively \textit{probably infected} and \textit{probably not infected}.

% \subsubsection{Confusion matrix}
% We define a confusion matrix:
% \begin{center}
%     CM(\bm{L}, f, th) = 
%     \begin{tabular}{c|cc}
%     &  $d(\bm{p}_i)=0$ & $D(\bm{p}_i)=1$ \\
%         $d(\bm{p}_i)=0$  & TN & FN \\
%         $d(\bm{p}_i)=1$ & FP & TP
%     \end{tabular}
% \end{center}

\begin{table}[b]
    \centering
    \begin{tabular}{lll}
    value & name & Usage \\
    NaN & & DNS packet is not a response \\
    0 & NOERROR & DNS Query completed successfully.  \\
    3 & NXDOMAIN & Domain name does not exist. \\
    Other & & Ignored \\ 
    \end{tabular}
    \caption{Possible DNS return codes.}
    \label{tab:rcodes}
\end{table}

% \subsubsection{Packet functions}


% \subsection{Pcap and windowing functions}


% \subsection{Detection function}



% \section{Results}

% \subsection{comparison different pre-processing methods}


% \subsection{comparison with nx function}


% \subsection{Using white list domain}

% \subsection{}


\section{Functions}

\subsection{Configuration parameters}


The initial configuration depends on these parameter:
\begin{itemize}
    \item $\nu$: neural network;
    \item $(t_1, t_2)$: white listing mode;
    \item $(n, p)$: replacing infinite values method;
    \item $s$: window size;
    \item $w_t$: windowing type.
\end{itemize}

\subsubsection{Neural Network (nn)}
It indicates which neural network we will use. The available neural networks are all trained with the same dataset, but each of them differ on how we handle the suffix.
We have four neural networks:
\begin{itemize}
    \item \textit{none}: it does not remove anything from the domain name.
    \item \textit{tld}: it removes only the rightmost label.
    \item \textit{icann}: it removes the longest ICANN suffixes (co.uk, co.cn). If no ICANN domain are present, it removes the TLD.
    \item \textit{private}: it removes the private suffix, like \texttt{compute.amazonaws.com.cn}. If there is no private suffix, it removes the icann suffix \texttt{com.cn}, if no icann suffix is available it removes just the tld \texttt{cn}.
\end{itemize}
\begin{equation}
    \nu \in \left\{ \text{none}, \text{tld}, \text{icann}, \text{private}  \right\}
\end{equation}


\subsubsection{White listing}
This parameter is a tuple of number: $(t1, t2)$. It neutralizes those domain names that belongs to the top 10 million domain list having ranking lower than $t1$, by assigning them a value equal to $t2$.
\begin{equation}
    \left\{ (t_1, t_2) \: | \: t_1 \in [0, 10^7],\:t2 \in \mathbb{R} \right\}
\end{equation}


\subsubsection{Infinite values}
When calculating the $\lambda$ value of a domain name inference, we may occur in infinite values if $\varepsilon$ is equal to $0$ or $1$. In those cases we have to substitute these \textit{not-a-number} values with some finite one.\\
This parameter is a tuple of numbers $(n, p)$, where $n$ substitutes the $-\infty$ and $p$ substitutes $\infty$.
\begin{equation}
    (n, p)
\end{equation}

\subsubsection{Window size}
It defines the size of the window used to split the pcaps.
The value chosen are:
\begin{equation}
    s \in \left\{ 100, 500, 2500 \right\}
\end{equation}

\subsubsection{Window type}

\label{sec:window-type}

It defines what DNS packet will be included in the analysis:
\begin{itemize}
    \item $w_t=w_t^q$, \textit{request}: considering only the requests.
    \item $w_t=w_t^r$, \textit{response}:  considering only the responses.
    \item $w_t=w_t^qr$, \textit{both}: considering both.
\end{itemize}
In order to compare two different configurations correctly, having the same window size, the DNS packet belonging to a window are defined, in any case, by the number of requests. Saying that $w_l$ is the number of DNS packets belonging to a window:
\begin{itemize}
    \item if we choose $w_t=w_t^q$, then $w_l$ will be equal to $s$.
    \item if we choose $w_t=w_t^r$, then $w_l$ will be equal to the number of responses included between the first and the last DNS packet of the \textit{request} window. In that case it could be greater or lower than $s$.
    \item if we choose $w_t=w_t^{qr}$, then $w_l$ will be equal to the number of responses included between the first and the last DNS packet of the \textit{request} window, plus $s$, the number of requests.
\end{itemize}

Defining $d^x_i$ the $i$-th DNS packet belonging to a pcap, where $x \in \{ q, r \}$ denotes if it is a request or response:
\begin{equation}
    \mathbf{l} = [ d^q_0, d^q_1, d^r_2, d^q_3, d^q_4, d^r_5, d^r_6, d^q_7, d^r_8, d^q_9, d^q_{10} ]
\end{equation}
then the first window $w_0$ with $s=5$ will be:
\begin{align}
    w_0\bigg\rvert_{w_t=w_t^q} &= [ d^q_0, d^q_1, d^q_3, d^q_4, d^q_7 ] \\
    w_0\bigg\rvert_{w_t=w_t^r} &= [ d^r_2, d^r_5, d^r_6 ] \\
    w_0\bigg\rvert_{w_t=w_t^{qr}} &= [ d^q_0, d^q_1, d^r_2, d^q_3, d^q_4, d^r_5, d^r_6, d^q_7 ] \\
\end{align}

\subsubsection{Configuration}

Each test is now dependant on a set of parameters that we call \textit{configuration}.
The parameters of a configuration are:
\begin{equation}
    \mathbf{c} = (\nu, (t_1, t_2), (n, p), s, w_t)
\end{equation}


\subsection{Function f}
Function f is the first function we apply to our test dataset.
\begin{equation}
    f(\mathbf{c}) = \mathbf{F}
\end{equation}

where $\mathbf{F}$ is a list of window processing. Its length is the total number of windows belonging to the test dataset.
Each window is defined by:
\begin{itemize}
    \item $w_l$: number of packets (req or res) included in the window. It may differ from $s$ because $s$ defines the number of requests, but if we consider also responses this number may differ.
    \item $w_{\lambda}$: the $\lambda$ value of the window.
    \item $w_n$: the sequential number of the window, in the order it appears in its parent pcap.
    \item $w_\omega$: the vdga value of the window inherited from its parent pcap.
    \item $w_p$: the parent pcap id.
\end{itemize}

\begin{table}[H]
    \centering
    \begin{tabular}{lllll}
        \toprule
        $w_l$ & $w_\lambda$ & $w_n$ & $w_\omega$ & $w_p$ \\
        \midrule
        ... & ... & ... & ... & ... \\
        4998 & -153538.42 & 226 & 2 & 845 \\
        5000 & -150450.49 & 227 & 2 & 845 \\
        5000 & -153332.59 & 228 & 2 & 845 \\
        4864 & -147630.55 & 229 & 2 & 845 \\
        305  &   -8539.18 & 230 & 2 & 845 \\
        4024 &   68938.46 &  0  & 3 & 846 \\
        3834 &   69434.35 &  1  & 3 & 846 \\
        3904 &   71129.25 &  2  & 3 & 846 \\
        3629 &   65489.62 &  3  & 3 & 846 \\
        3719 &   67523.31 &  4  & 3 & 846 \\
        4824 &   90093.52 &  5  & 3 & 846 \\
        ... & ... & ... & ... & ... \\
        \bottomrule
    \end{tabular}
    \caption{Example of function $f$ output $\mathbf{F}$.}
\end{table}


\subsection[Function l]{Function $l$}
Function l is the second function having as input the output of function f and a $n_{th}$ parameter:
\begin{equation}
    l(\mathbf{F}, n_{th}) = l(f(\mathbf{c}), n_{th}) = \mathbf{L}
\end{equation}

The $n_{th}$ parameter defines the number of thresholds that we use to test each window in $\mathbf{F}$. If $w_{\lambda}^{m}$ and $w_{\lambda}^{M}$ are respectively the minimum and maximum value of $w_i^{\lambda} \in \mathbf{F}$, then:
\begin{equation}
    th_j = w_{\lambda}^{m} + th_{step} \cdot j
\end{equation}
where
\begin{equation}
    th_{step} = \frac{w_{\lambda}^{M} - w_{\lambda}^{m}}{n_{th}-1}
\end{equation}

Given a $th_j$, we calculate the confusion matrix:
\begin{align}
    \text{tn}_j &= \sum_i \begin{cases}
    1 & \text{if } w_i^{\lambda} > th_j,\: w_i^{\omega} = 0\\
    0 & \text{otherwise}
    \end{cases} \\
    \text{fp}_j &= \sum_i \begin{cases}
    1 & \text{if } w_i^{\lambda} > th_j,\: w_i^{\omega} = 0\\
    0 & \text{otherwise}
    \end{cases} \\
    \text{fn}^{\omega'}_j &= \sum_i \begin{cases}
    1 & \text{if } w_i^{\lambda} > th_j,\: w_i^{\omega} \ne \omega'\\
    0 & \text{otherwise}
    \end{cases} \\
    \text{tp}^{\omega'}_j &= \sum_i \begin{cases}
    1 & \text{if } w_i^{\lambda} > th_j,\: w_i^{\omega} = \omega'\\
    0 & \text{otherwise}
    \end{cases} \\
\end{align}
where $\omega' \in { 0,1,2,3} $.
Finally for each windows we have:
\begin{align}
    cm_i = \left( \begin{array}{cc}
        \text{tn}_i & \text{fp}_i \\
        \text{fn}^{\omega'}_i & \text{tp}^{\omega'}_i
    \end{array}
    \right)
    \label{eq:confusion-matrix}
\end{align}
where:
\begin{align}
    \text{fn}^{\omega'}_i &= \{ \text{fn}^0, \text{fn}^1, \text{fn}^2, \text{fn}^3 \} \\
    \text{tp}^{\omega'}_i &= \{ \text{tp}^0, \text{tp}^1, \text{tp}^2, \text{tp}^3 \}
\end{align}



The output $\mathbf{L}$ is a list of $n_{th}$ confusion matrix as defined in \ref{eq:confusion-matrix}:

\begin{table}[H]
    \centering
    \begin{tabular}{rrrll}
    \toprule
    {} & tn & fp & fn & tp \\
    threshold &&&& \\
    \midrule
    -29244.79 &    0 &  126 &             [1, 0, 0, 1] &  [5240, 2210, 611, 2419] \\
    -28848.94 &    0 &  126 &             [1, 0, 0, 1] &  [5240, 2210, 611, 2419] \\
    -28453.08 &    1 &  125 &             [1, 0, 0, 1] &  [5240, 2210, 611, 2419] \\
    ... & ... & ... & ... & ... \\
     -9452.19 &   123 &   3 &    [2681, 2075, 585, 21] &    [2560, 135, 26, 2399] \\
     -9056.34 &   124 &   2 &    [2682, 2075, 585, 22] &    [2559, 135, 26, 2398] \\
     -8660.49 &   124 &   2 &    [2683, 2075, 585, 23] &    [2558, 135, 26, 2397] \\
    ... & ... & ... & ... & ... \\
     48738.03 &  126 &    0 &  [4801, 2160, 609, 1806] &        [440, 50, 2, 614] \\
     49133.89 &  126 &    0 &  [5193, 2197, 611, 2385] &          [48, 13, 0, 35] \\
     49529.74 &  126 &    0 &  [5238, 2207, 611, 2420] &             [3, 3, 0, 0] \\
    \bottomrule
    \end{tabular}
    \caption{Example of function $f$ output $\mathbf{F}$.}
    \label{tab:function-l}
\end{table}

\subsection[Function p]{Function $p$}

Function p is the second function having as input the output of function $l$ and the $\mathbf{th_2}$ parameter. This parameter is an array of thresholds, called \textit{threshold of type 2}, to distinguish them from the thresholds used in function l.

\begin{equation}
    p(\mathbf{L}, \mathbf{th_2}) = \left\{ m_r^{th_2},\: \forall th_2 \in \mathbf{th_2} \right\} = \mathbf{P}
\end{equation}

where $m_r^{th_2}$ is called range\textit{ metric}: it depends on threshold of type-2, $th_2$ and it defines the maximum contiguous domain portion within the following condition is satisfied:
\begin{equation}
    \text{tp} > th_2 \: \wedge \: \text{tp} > th_2
\end{equation}
%
For example, see at table \ref{tab:function-l-relative} and at figure \ref{fig:th2}.

\begin{table}[H]
    \centering
    \begin{tabular}{rllll}
    \toprule
          ths &   tn &                       tp & $m_r^{\sfrac{1}{4}}$ & $m_r^{\sfrac{1}{2}}$ \\
    \midrule
    $-29244.79$ & $0.00$ & $[1.00, 1.00, 1.00, 1.00]$ & $[0,0,0,0]$ & $[0,0,0,0]$ \\
    $-28848.94$ & $0.00$ & $[1.00, 1.00, 1.00, 1.00]$ & $[0,0,0,0]$ & $[0,0,0,0]$ \\
    $-28453.08$ & $0.01$ & $[1.00, 1.00, 1.00, 1.00]$ & $[0,0,0,0]$ & $[0,0,0,0]$ \\
            ... &    ... &                        ... &   ... \\
     $-9452.19$ & $0.98$ & $[0.49, 0.06, 0.04, 0.99]$ & $[1,0,0,1]$ & $[0,0,0,1]$ \\
     $-9056.34$ & $0.98$ & $[0.49, 0.06, 0.04, 0.99]$ & $[1,0,0,1]$ & $[0,0,0,1]$ \\
     $-8660.49$ & $0.98$ & $[0.49, 0.06, 0.04, 0.99]$ & $[1,0,0,1]$ & $[0,0,0,1]$ \\
            ... &    ... &                        ... &   ... \\
     $48738.04$ & $1.00$ & $[0.08, 0.02, 0.00, 0.25]$ & $[0,0,0,1]$ & $[0,0,0,0]$ \\
     $49133.89$ & $1.00$ & $[0.01, 0.01, 0.00, 0.01]$ & $[0,0,0,0]$ & $[0,0,0,0]$ \\
     $49529.74$ & $1.00$ & $[0.00, 0.00, 0.00, 0.00]$ & $[0,0,0,0]$ & $[0,0,0,0]$ \\
     \bottomrule
    \end{tabular}
    \caption{In reference to table \ref{tab:function-l}, this table shows relative values of tn and tp.}
    \label{tab:function-l-relative}
\end{table}


\begin{figure}
    \centering
    \includegraphics[width=\textwidth]{figures/plot_th2.png}
    \vspace*{-10mm}
    \caption{The red and blue plot are respectively the relative true positive and negative values for each $th$. $m_r*$ represent the renge metric values for $th_2 \in \{ 0.25, 0.5, 0.75 \}$.}
    
    \label{fig:th2}
\end{figure}


\paragraph{Function p output} It calculates some metrics about the result of function l:
\begin{itemize}
    \item Sum metric, $m_s$: the maximum value of $\text{tn} + \text{tp}$.
    \item Range metric, $m_r(th_2)$: it depends on threshold of type-2, $th_2$, and it is the maximum number of \textbf{consecutive} thresholds of type-1 which satisfies $(\text{tp} > th_2 \wedge \text{tp} > th_2)$.
\end{itemize}
We can see these metrics in Figure \ref{fig:function-p-metric} where the blue and red plots are respectively the true negatives and true positives calculated over $n_{th}=200$ thresholds (x-axis). The \textit{sum metric} is represented by the blue and red scatters, while the \textit{range metric} is represented by the horizontal arrows, one for each $th_2 \in \{0.25, 0.50, 0.75 \}$.
These two metrics describe the best choice possible we can do for the threshold. In a practical environment, we have to choose some metric upon we instruct our detector to generate alarms. The \textit{sum metric} tell us the best possible result we can obtain but it is limited to our testing environment. Otherwise the \textit{range metric} tell us the range between our functions perform better.
If we look at Table \ref{tab:function-p-sum-range-metrics} and Figure \ref{fig:function-p-metric}. We can conclude the following:
\begin{itemize}
    \item limiting the studied windows to specific $\omega'$ changes conspicuously the \textit{range metric}.
    \item we obtain very good predictions with at least one threshold.
    \item the number of thresholds within we have very good predictions is very little for certain $\omega$, indeed the performance of the detector could be affected by some bias and the probability to have very bad performance if we choose the wrong threshold is higher if $m_r$ is smaller.
    \item for $\omega=\omega^3$ we have very good results.
\end{itemize}

\begin{table}[H]
    \centering
    \begin{tabular}{@{}lllllll@{}}
    & \phantom{a} &  \phantom{a} &&&& \\
    $\omega$     && \multicolumn{1}{c}{$m_s$} && $m_r^{\sfrac{1}{4}}$ & $m_r^{\sfrac{1}{2}}$ &  $m_r^{\sfrac{3}{4}}$ \\
    \cmidrule{1-1}
    \cmidrule{3-3}
    \cmidrule{5-7}
    $\omega*$    &&     0.89, 0.94 &&   0.66 &   0.16 &   0.09  \\
    $\omega^{1}$ &&     0.88, 0.97 &&   0.13 &   0.11 &   0.07  \\
    $\omega^{2}$ &&     0.88, 0.63 &&   0.10 &   0.09 &   --    \\
    $\omega^{3}$ &&     1.00, 0.99 &&   0.68 &   0.65 &   0.65  \\
    \end{tabular}
    \caption{Caption}
    \label{tab:function-p-sum-range-metrics}
\end{table}


\begin{figure}[H]
    \centering
    \includegraphics[width=\textwidth]{figures/plot_metric_all.png}
    \caption{}
    \label{fig:function-p-metric}
\end{figure}



\section{Compare configurations}

\subsection{Parameter analysis}

\subsubsection{Neural network}




\subsubsection{White listing}
\subsubsection{Infinite values}
\subsubsection{Window size}
\subsubsection{Window type}





\bibliographystyle{alpha}
\bibliography{sample}

\end{document}